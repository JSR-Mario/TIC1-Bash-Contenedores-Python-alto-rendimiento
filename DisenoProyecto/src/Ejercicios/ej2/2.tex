\textbf{¿Por qué este proyecto es relevante?}\vspace{.3cm}

Ya he hablado un poco sobre por qué considero que mi proyecto podría ser relevante. De manera concisa, este proyecto puede abrirte las puertas para analizar tus propios datos y comportamiento; después de todo, es en nuestras relaciones donde se refleja quiénes somos. \vspace{.3cm}

\textbf{¿Para quién estás trabajando?}\vspace{.3cm}

\begin{enumerate}
    \item \textbf{¿Qué quieres lograr con tu proyecto? ¿Qué problema o situación estás abordando?}
    
    Quiero poder responder dudas acerca de mis conversaciones de WhatsApp de manera rápida y automatizada, así como identificar patrones en mi comportamiento o en el de mis relaciones.

    \item \textbf{¿Para quién estás programando? ¿Quién es el usuario final?} 
    
    Principalmente para mí, pero si el resultado es suficientemente bueno, podría compartirlo con quien le interese :)

    \item \textbf{¿Qué decisiones necesita tomar el programa para funcionar?}
    
    Lo más importante es que verifique que las entradas y las banderas sean válidas. Fuera de eso, el programa simplemente actúa en función de los datos proporcionados.

    \item \textbf{¿De qué manera se puede romper el programa? ¿Cómo puedes prevenir que el usuario hackee tu programa? ¿Cómo puedes validar los datos que te brinda el usuario?}
    
    En general, se deben validar los formatos de entrada y asegurarse de que las banderas sean correctas. Evidentemente, siempre habrá formas de romper el programa en términos lógicos, pero este no debería devolver más que excepciones controladas y debe manejar los errores de manera limpia.

    \item \textbf{¿Cómo puedes prevenir los sesgos (racismo, sexismo, colonialismo, discriminación a personas con discapacidades, etc.) en tu proyecto?}
    
    El programa únicamente mostrará métricas del chat; no habrá procesamiento de contenido más allá de eso. Sin embargo, su uso podría permitir identificar y analizar comportamientos sesgados en nuestras propias interacciones.
\end{enumerate}

